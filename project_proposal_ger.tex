\documentclass[a4paper,12pt]{article}
\pagestyle{headings}
\usepackage[utf8]{inputenc}
\usepackage{aeguill} 
\usepackage{wrapfig}
\usepackage{graphicx}
\usepackage{gensymb}
\usepackage{fullpage}
\usepackage[ngerman]{babel}
\graphicspath{ {project_proposal_img/} }
 

\begin{document}

\begin{titlepage}
\begin{center}

% Upper part of the page. The '~' is needed because \\
% only works if a paragraph has started.
\includegraphics[width=0.35\textwidth]{patch}~\\[2cm]

\textsc{\Large Projektvorschlag}\\[0.5cm]

% Title
\huge \bfseries Octanis 1: Ein autonomer low-cost Roboter für die Antarktis mit satelitengestützter Datenübertragung \\[0.4cm] 

\vspace{23pt}
\includegraphics[width=0.25\textwidth]{black_logo} \\
\vfill

% Bottom of the page
{\large \today} \\
\textsc{\small info@octanis.org | www.octanis.org}
\vspace{50pt}


\begin{table}[h!]
\centering
\vspace{1pt}
\begin{tabular}{ l  l  l }
	\textbf{Revisionen} & \textbf{Änderungen} & \textbf{Autoren} \\
	25.6.14 & Englische Originalfassung & Sam Sulaimanov, Raffael Tschui, \\ & & Ana Roldán, Pamela Canjura \\
	27.6.14 & Spanische Verison & Ana Roldán \\
	27.6.14 & Deutsche Version & Raffael Tschui \\
\end{tabular}
\end{table}

\end{center}
\end{titlepage}

\tableofcontents

\pagebreak

\section{Einführung}
»octanis | discovery and exploration« \cite{octanis} ist ein ambitiöses und herausforderndes Projekt, initiiert durch eine interdisziplinäre Gruppe von Studenten. Octanis 1 ist unsere erste Mission mit dem Ziel, einen günstigen, autonomen Roboter zu bauen der sich in polaren, schnee- und eisbedeckten Regionen bewegen kann. In einem ersten Schritt spezialisieren wir den Roboter für Wetterbedingungen der küstennahen Gebiete der Antarktis. Der Roboter wird Daten wie Temperatur, Luftdruck, relative Luftfeuchtigkeit, aktuelle Position und Lage übertragen können. Optische Sensoren an Bord werden Hindernisse erkennen und Fotos der näheren Umgebung machen können. Weitere spezifischere Datensammlung ist durchaus möglich, muss aber wegen der Energie- und Platzbeschränkungen speziell geprüft werden. Beispiele dafür sind die Konzentration eines bestimmten Gases in der Atmosphäre, $\alpha, \beta, \gamma$ Strahlung oder die Entnahme von Schnee- oder Eisproben. \\
Als eine geeignete erste wissenschaftliche Mission für Octanis 1 \cite{krishnakant} bestimmten wir die regelmässige Bodenentnahme von Schnee und Eis. Der Roboter wird mit einem kleinen holen Gerät ausgestattet sein, welches ihm ermöglicht, in die Oberfläche zu bohren, den Schnee oder das Eis zu schmelzen und schliesslich dessen pH-Wert bestimmen.\\ 
Die Energie für Octanis liefern die Solarzellen auf seiner Oberfläche, welche die Sonnenenergie in Strom umwandeln und in einer Batterie speichern können. Dank dem Irdium Satelitennetzwerk \cite{iridium}, welches eine günstige, energieeffiziente und zuverlässige Datenübertragung bereitstellt, ist es uns möglich, mit dem Roboter in einem beliebigen Punkt der Erde zu kommunizieren.


\section{Überblick}

Das Ziel der Mission Octanis 1 ist eine günstige Roboterplatform zur Verfügung zu stellen, welche nur kleine Auswirkungen auf die Umwelt hat und sich für wissenschaftliche Experimente in einer sehr kalten Umgebung eignet. Der Roboter wird so klein und leicht sein, dass er mit einem Wetterballon transportiert werden kann und seine Bauweise ermöglicht es ihm, eisiges Terrain zu durchqueren und gleichzeitig windresistent zu sein. Dank Solarzellen wird er alle nötige Energie selbst generieren und kann somit seine interne Temperatur regulieren, was sich als überlebenswichtig herausstellt. Die Platform des Roboters bekommt vier Räder, die allesamt ????????????????????????... being on a controllable strut, allowing it to drive in any orientation and right itself should it flip over. 


\subsection{Ziele}

\paragraph{Robotik}
Es soll ein allwettertauglicher Leichtbauroboter entstehen mit einem einfachen, aber robusten Hardware- und Softwaredesign. Der Roboter muss die Mission grösstenteils autonom zu Ende führen können mit Hilfe nur da wo sie nötig ist. Eine Mission enthält einen abzufahrenden Pfad und auszuführende Messungen. Über das Internet können Abbruch- oder Hilfskommandos zum Roboter gesendet werden. Das interne Kraftwerk soll dem Roboter genügend Energie liefern damit er eine Mission während mehrerer Monate ohne menschliches Zutun ausführen kann.

\paragraph{Sensorik}
An Bord des Roboters befinden sich Messgeräte, welche regelmässig den Status der inneren und äusseren Umgebung speichern und übertragen können. Die standardisierten Schnittstellen werden es ermöglichen, wissenschaftliche Instrumente einfach ein- und auszubauen. Sämtliche übertragene Daten von und zum Roboter werden öffentlich in Echtzeit im Internet abrufbar sein. Eine spezifische Schnittstelle wir des ihm erlauben, mit zukünftigen Roboter in einem Sensornetzwerk zu kommunizieren und als Schwarm funktionnieren.

\paragraph{>>>>>>>>> gutes Deutsches Wort für Deployment. (Landung, Abwurf, Stationierung)?}
Es stehen verschiedene Möglichkeiten zur Verfügung, den Roboter in die Antarktis zu befördern, deren Kosten und Risiken noch abzuwägen sind.



\paragraph{Offen \& Reproduzierbar} 
Wir bauen Octanis in Anlehnung an die Open Source Bewegung. Sämtliche Dokomentation, Software und 3D-Modelle sind auf unserer Website zum Download verfügbar, so dass sie jede(r) wiederverwenden, verändern und verbessern kann. Wir verwenden, wo immer möglich, einfach zu beschaffende Bauteile damit jede(r) mit den nötigen Kenntnissen seine/ihre eigene Version von Octanis bauen kann. Von uns verwendete Fabrikationstechniken wie 3D-Druck sind mehr und mehr verbreitet und stehen in sogenannten Fab Labs \cite{fablab} oder Hackerspace \cite{hackerspace} einem breiten Publikum für wenig Geld zur Verfügung.
 

\subsection{Zeitplan}

Hier genannte Daten gelten für das Jahr 2014 falls nicht anders deklariert. Die genauen Daten für die wissenschaftliche Mission werden noch in Übereinstimmung mit der kooperierenden Antarktisbasis bestimmt. Das Projektmanagement wird transparent und agil geführt und der Entwicklungsprozess ist sehr zyklisch damit kleine Änderungen sofort mit eingebaut werden können wenn sie nötig werden. Diese Methode konnte sich in der Vergangenheit als erfolgreich beweisen weil Fehler schnell korrigiert und die Kosten tief gehalten werden konnten. Transparenz wird dadurch geschaffen, dass die gesamte Dokumentation auf unseren öffentlichen GitHub \cite{octanisgithub} Repositories vorhanden ist.

\begin{table}[h!]
\centering
\begin{tabular}{ l | l | l | c }

\bfseries{Code} & \bfseries{Projektphase} & \bfseries{Daten} & \bfseries{Wochen} \\
\hline
A1 & Prototyp Design \& Entwicklung & 1.6. - 1.10. & 12 \\
A2 & Software Entwicklung & 1.6. - 1.10. & 12 \\
B1 & Erste Tests & 1.10. - 14.10. & 2  \\
B2 & Testfahrt \& Bohrtests auf Schweizer Gletscher & 14.10. - 14.11. & 4 \\
D & Mechanische Stresstests & 14.11. - 21.12. & 1 \\
E & Tests Energiemanagement & 21.11. - 14.12. & 3 \\
F & Fallschirmtests & 14.12. - 21.12. & 1 \\
0.1 & Abwurfmöglichkeit in der Antarktis & 14.2.2015 &  1 \\
0.2 & Abwurfmöglichkeit in der Antarktis & 1.12.2015 &  1 \\

\end{tabular}
\caption{Zeitplan Entwicklungsphasen}
\end{table}

Nach dem Abwurf des Roboters in die Antarktis wird das Wissenschaftsprojekt wie in der folgenden Tabelle beschrieben weitergeführt. Zu beachten ist, dass während der gesamten Mission eine Satelitenverbindung zum Roboter besteht, womit mehrmals täglich Statusinformationen und Messdaten übertragen werden. Um den Verlauf der Mission zu beeinflussen können jederzeit Kommandos gesendet werden.


\begin{table}[h!]
\centering
\begin{tabular}{ l | l | c }
\bfseries{Code} & \bfseries{Projektphase} & \bfseries{Wochen} \\
\hline

0 & Abwurf & 1 \\
1 & Check der Systeme & 1 \\
2 & Fahrt und Probeentnahme & 3 \\
3 & Kameratests \& Bildübertragung & 3 \\
4 (*)& Test der Amateurfunk-Datenübertragung  & 3 \\

\end{tabular}
\caption{Zeitplan Wissenschaftsprojekt}

*Je nach Abwurfort des Roboters könnte Funkkontakt zur Antarktisbasis bestehen. 

\end{table}

\pagebreak

\subsection{Team}

Wir sind eine Gruppe Studenten, die ihre Fähigkeiten bis an ihre Grenzen herausfordern. Octanis 1 wird nicht nur von uns gebaut werden, sondern mit der Hilfe vieler Professoren und Ratgebern, die wir im Verlauf des Projektes antreffen werden. Die nachfolgende Präsentation stellt deshalb das Kernteam vor:




\paragraph{Sam Sulaimanov} 
\begin{wrapfigure}{l}{0.2\textwidth}
    \centering
    \vspace{-13pt}
    \includegraphics[width=0.15\textwidth]{sam}
\end{wrapfigure} besitzt sieben Jahre Arbeitserfahrung als Programmierer und Kommunikationsnetzwerkingenieur. Seit jungen Jahren tüftelt er mit Microcontrollern und Elektronik herum und wird sicherstellen, dass Octanis' Gehirn korrekt funktionniert.
\\ \\

\begin{wrapfigure}{l}{0.2\textwidth}
    \centering
    \vspace{-13pt}
    \includegraphics[width=0.15\textwidth]{ana}
\end{wrapfigure}
\paragraph{Ana Roldàn} ist eine leidenschaftliche angehende Physikerin und in allen Bereichen des Projektes involviert. Sie ist nie zu scheu, die grossen Fragen zu stellen, und inspiriert alle anderen mit ihrer grossen Leidenschaft für die Wissenschaft.
\\ \\

\begin{wrapfigure}{l}{0.2\textwidth}
     \centering
     \vspace{-13pt}
    \includegraphics[width=0.15\textwidth]{raf}
\end{wrapfigure} 
\paragraph{Raffael Tschui} besitzt einen Bachelor in Elektrotechnik der EPFL und kümmert sich um die lebenswichtige Energieversorgung und  -verwaltung des Roboters. Im Moment arbeitet er zusammen mit einem EPFL Professor an einem Projekt in Kolumbien, in dem er beim Bau eines wasserreinigenden Bioreaktor mitwirkt.  
\\ \\

\begin{wrapfigure}{l}{0.2\textwidth}
    \centering
    \vspace{-13pt}
    \includegraphics[width=0.15\textwidth]{pam}
\end{wrapfigure} 
\paragraph{Pamela Canjura} interessiert sich seit sie fünf Jahre alt war für Chemie und nimmt regelmässig an der Chemie Olympiade teil. Sie befasst sich mit den chemischen Analysen, welche an Bord von Octanis gemacht werden können.
\\ \\


\subsection{Budget}

Tabelle \ref{tabbudget} listet die geplanten Kosten auf mit einer Toleranz von $\pm 20\%$ (krit. = Missionskritisch). \\ 

\begin{table}[h!]
\centering
\begin{tabular}{ l | c || r }
  Beschreibung & Priorität & Einmalige Kosten \\
  \hline
  Roboter Solar \& Heizung & krit. & 1000 CHF \\
  Roboter Kommunikation & krit. & 500 CHF \\
  Roboter Instrumente \& Sensoren & krit. & 500 CHF \\
  Roboter Elektronik & krit. & 700 CHF \\
  Roboter Mechanik \& Mobilität & krit. & 1000 CHF \\
  Ballon Transport & opt. & 1500 CHF \\
  \hline \hline
  & & kritisch: 3700 CHF  \\
  & & optional: 1500 CHF \\
\end{tabular}
\label{tabbudget}
\caption{Allgemeines Budget.}
\end{table}

Die einzigen wiederkehrenden Kosten werden von der Irdium Satelitenübertragung \cite{iridium} verursacht. Diese reichen von 0.04-0.12 GBP pro Nachricht (340 bytes) zuzüglich der Verbindungsmiete von 8 GBP pro Monat. Der Preis einer Nachricht hängt von der Anzahl benötigten Nachrichten ab.



\section{Rover Transportation and Deployment}

Octanis 1 is a light-weight rover meaning its total mass will not exceed 2.5kg. Thanks to this there are multiple cost-effective deployment options available which are listed below:

\subsection{Methods of deployment}

\begin{figure}[h!]
	\centering
    \includegraphics[width=0.4\textwidth]{trajectory}
    \caption{HYSPLIT trajectory simulation starting in Rio Grande, Argentina.}
\end{figure}

\paragraph{High Altitude Balloons} or more commonly known, weather balloons, can be used to bring the rover to its target destination. Helium-filled, they typically float up to an altitude of 30km and a trajectory of hundreds of kilometers can be achieved. However the path of the balloon cannot be controlled and relies on rigorous pre-flight simulations with the HYSPLIT simulation software \cite{hysplit} \cite{hysplitjava}. This is the cheapest method to deploy the rover to Antarctica as it is only required to travel to lower South America, to which commercial flights exist, to launch the rover attached to the balloon. With this benefit comes a higher risk of deployment failure. Even with precise weather models, the weather could suddenly change and move the balloon off its path. That being said, HYSPLIT was already successfully used in numerous ballooning projects like Piccard and Jones' Breitling-Orbiter 3 \cite{hysplitexamples}.
Once the balloon-rover-configuration have reached their target destination, the rover is separated from the balloon and parachutes to the ground. The balloon continues on uncontrolled and will eventually burst or float to the ground after the helium diffuses out.

\begin{figure}[h!]
	\centering
    \includegraphics[width=0.5\textwidth]{lowcostchute}
    \caption{Depicted is one of our self-built, low-cost parachutes during a test. }
\end{figure}


\paragraph{Helicopter} - a method to consider is to deploy Octanis out of a helicopter which is already on a specific route. The rover could be given to one of the passengers or attached to the transport cord and detached on demand. Similar to balloon deployment, Octanis would swiftly parachute to the ground and begin its mission.


\paragraph{Manual} deployment is suggested when costs and risk need to be as low as possible. This is by far the easiest method and just means going to the desired location and setting the rover down. Octanis continues on by its own and can return to the set-down location.




\subsection{Environmental Sustainability}
It is desirable to not leave any waste behind as well as it is required by the Antarctic Treaty. Depending on the method of deployment, the total mass and types of materials that will be included on a mission to Antarctica vary. A deployment of the rover to a target out of reach of normal Antarctic expeditions is done by using a High Altitude Balloon or a standard 3kg latex weather balloon. In such a long-distance mission, the rover will be separated from the balloon at the target location and the balloon will continue to fly unattended. Therefore the landing location of the balloon can not be known. Due to the nature of such a long-distance mission, typically the rover is unreachable to any expedition or base - the rover is left for the benefit of information on that location.

In a first step, it is therefore proposed to select a landing site that is in reach of normal Antarctic expeditions or bases. It is even possible to release the rover by other means like bringing it to the location manually or via helicopter.

Concluding, the rover is a construction of various polymers and metals, and a small risk exists that it will end up being uncontrollable due to malfunction. The total rover mass however does not exceed 2.5kg and therefore the pollution produced is small. In the unlikely event of failure, the rover can then be retrieved manually thanks to the last known transmitted location.




\section{Roboter Subsysteme}

\subsection{Board Computer}
\begin{figure}[h!]
	\centering
    \includegraphics[width=1\textwidth]{schema}
    \caption{Schema des Rechnersystems und der Peripherie an Bord}
\end{figure} 

Die 16-bit MSP430 Mikrocontrollerfamilie von Texas Instruments ist die bevorzugte Wahl für das Computersystem im Roboter. Der Mikrocontroller übernimmt die meisten Kontroll- und Regelungsaufgaben, insbesondere Fahren, Navigieren, Kommunikation und Bearbeiten von Sensordaten. Darauf laufen wird ein Realtime Operating System, welches mit Multitasking unterschiedlich priorisierte Aufgaben zuverlässig ausführen kann. Der Computer interagiert mit den folgenden Peripheriegeräte an Bord:

\begin{itemize}
\item Interne Messgeräte: Beschleunigungssensor, Gyroskop, Magnetometer, Präzisionsbarometer.
\item Wettermodul: Temperatur, Barometer, Hygrometer, Anemometer (Winddruckmesser).
\item Wissenschaftliches Instrument: pH-Sonde.
\item Optisches Modul: Kamera, Distanzsensor.
\item Antriebsmodul: Motorensteuerung, Stromflussmesser.
\item Kommunikation: Iridium, APRS, ZigBee.
\item SD-Speicherkarte.
\item GPS Modul.
\end{itemize}

Falls der Hauptcomputer ausfällt, würde ein zusätzlicher Hilfscomputer das Kommando übernehmen. Dieser würde versuchen, den Hauptcomputer neu zu starten, oder wenn dies nicht gelingt, das System mit seinen Grundfunktionen am Laufen zu halten.


\subsection{Energieversorgung}

Die Sonnenenergie, die pro Quadratmeter auf die Erde trifft, ist $E_0*sin(\alpha)$ with $E_0=1367 W/m^2$ \cite{solarc} und $\alpha$ der Winkel zwischen den eintreffenden Sonnenstrahlen und der Horizontalen (siehe Abbildung \ref{sunrayangle}). Für die Antarktis können wir $\alpha \approx 23.5+90+|\phi|$ verwenden, um eine Abschätzung der maximal eintreffenden Sonnenenergie (d.h. im Sommer) an einem bestimmten geographischen Punkt zu erhalten, wobei $\phi$ der Breitengradswinkel ist.


\begin{figure}[h!]
	\centering
    \includegraphics[width=0.6\textwidth]{sun}
    \caption{Veranschaulichung des Sonneneinstrahlungswinkel in einem bestimmten Breitengrad}
    \label{sunrayangle}
\end{figure}

Dies würde am nördlichsten Punkt der Antarktis (bei 65\degree S) $1050 W/m^2$ und am Südpol (90\degree S) $545 W/m^2$ ergeben. Natürlich ist zu beachten, dass dieser Wert in den sechs Sommermonaten zwischen Null und dem eben berechneten Maximum schwankt \cite{pvedu}. Angenommen, unsere Panels sind horizontal ausgerichtet und unter Einbezug von aktuell gängigen Solarzellenwirkungsgraden (ca. 15-20\%) kann mit einer elektrischen Energieumwandlung von höchstens 200 W gerechnet werden. Mit einem eher schlechten Wirkungsgrad und einem geschätzten "Mittelwert" der Einstrahlung während den 6 Monaten der Mission (etwa die Hälfte des Maximumwertes) bräuchte es eine Solarzellenfläche von $0.13m^2$ (bei 65\degree S) bis zu $0.25m^2$ (bei 90\degree S) um 10 W elektrische Leistung zu erzielen. Diese Rechnung beinhaltet aber weder die tägliche Oszillation des Einstrahlungswinkel, noch die Tatsache, dass die Sonne in einigen Regionen zeitweise ganz verschwindet. Deshalb und auch weil die Intensität bei schlechtem Wetter abnimmt, sollte eher mit dem pessimistischen Wert von $0.25m^2$  gerechnet werden.

Zusammenfassend gesagt ist eine Solarzellenfläche von ungefähr $0.25m^2$ genug, um den Roboter mit durchschnittlich 10 W elektrische Leistung zu versorgen und diese Werte sind zueinander proportional.


\subsection{Heizung}

Höchste Priorität besteht darin, das System über -20\degree C zu halten, weshalb sämtliche Sonnenenergie und, wenn nötig, die Batterie zuerst für dieses Ziel verbraucht werden. Die Bordheizung ist deshalb direkt zwischen der Solarzelle und dem Batterienladegerät angeschlossen. Sie wird mittels elektronischen Komparatoren von der einen oder anderen Energiequelle betrieben, abhänging von der gemessenen Bordtemperatur. Der verwendete Batterieladeregler überwacht automatisch die empfohlenen Temperaturlimiten während des Aufladen und kann sich selbst an- und ausschalten.


\subsection{Mechanik}

Die Form des Roboters (Abb. \ref{cadmodel}) wurde nachhaltig durch die von der NASA entwickelten MUSES-CN Nanorover \cite{muses} beeinflusst. Das Design der Nanorover ist einzigartig und erlaubt eine fast uneigeschränkte Bewegungsfreiheit. Unser erster Entwurf von Octanis 1 ist leicht und besitzt eine genügend grosse Fläche um während der ganzen Mission ausreichend Solarenergie zu produzieren. Der Grundkörper (ohne Räder) misst ungefähr 30 cm x 30 cm 6 cm.

\begin{figure}[h!]
	\centering
    \includegraphics[width=0.5\textwidth]{conceptrover}
    \caption{Designkonzept für den Octanis Roboter, inspiriert durch MUSES-CN Nanorover}
	\label{cadmodel}
\end{figure}

Wie in Abbildung \ref{wheelconfig} gezeigt, kann sich der Roboter je nach Situation in unterschiedlicher Weise ausrichten und positionieren. Die Hauptprobleme der Mission werden Energieversorgung und Wetterfestigkeit sein, weshalb der ganze Roboter mit Solarzellen bedeckt ist und sich selbstständig auf- und je nach Sonnenstand ausrichten kann, so dass er ein Maximum seiner Fläche mit Sonnenlicht befluten lassen kann. Bei schlechtem Wetter muss sichergestellt sein, dass kein Schnee die Solarzellen bedeckt. Dies kann durch regelmässiges Umkehren des Roboters erreicht werden. \\
Nicht nur Schnee, sondern auch Wind kann Probleme verursachen und manchmal den Roboter sogar auf den Kopf stellen. In diesem Fall kann er sich selbstständig wieder in eine stabile, fahrbare Position bringen indem er seine Räderstrebungen bewegt. Alle Streben können um eine gemeinsame Achse gedreht werden und werden je von einem separaten Motor betrieben. Um in Ruhelage keinen Strom zu verbrauchen, sind sie mit einem Schneckengetriebe an den Motor gekoppelt. \\
Ein weiterer Vorteil dieses Designs besteht darin, dass der Einbau des Bohrwerkzeug vereinfacht wird. Beim Bohren einer Eisprobe senkt sich der Roboter bloss zum Boden herab und nutzt so sein eigenes Gewicht, um Druck auf den Bohrer auszuüben. Der Bohrkopf brauch nur zu rotieren und muss nicht in den Roboter hinein geführt werden.

\begin{figure}[h!]
\centering
\begin{tabular}{ c  c  c }
\includegraphics[width=0.3\textwidth]{drive} & \includegraphics[width=0.25\textwidth]{upright} & \includegraphics[width=0.4\textwidth]{flat} \\
\end{tabular}
\caption{Verschiedene Radstrebenpositionen v.l.n.r.: (1) Fahrmodus, (2) Solarlademodus, (3) Windschutzmodus.}
	\label{wheelconfig}
\end{figure}


\subsection{Mobilität}
Verschneites und eisiges Terrain wird Octanis' typisches Zuhause sein und deshalb müssen seine Räder speziell breit, leicht und mit einer grossen Aufwärts-Zugkraft konzipiert werden. Alle vier Räder sind separat motorisiert und erbringen ein genügend grosses Drehmoment, um Octanis eine Steigung von 30 Grad bewältigen zu lassen. Eine Drehung des Roboters kann durch unterschiedliche Rotationsgeschwindigkeiten der Räder auf beiden Seiten erreicht werden.

\begin{figure}[h!]
	\centering
    \includegraphics[width=0.4\textwidth]{wheel}
    \caption{Initial wheel design. Fins allow the wheel to dig into the snow while driving providing better traction.}
\end{figure}

Ein Inertiale Messeinheit (engl. inertial measurement unit, IMU) wird verwendet, um jederzeit die Ausrichtung des Roboters zu kennen. Sie besteht aus einem Gyroskop, Beschleunigungssensor, Präzisionsbarometer und einem Magnetometer. Die Sensordaten werden dem Kontrollsystem mitgeteilt welches die Motoren steuert. Falls der Roboter ein Hindernis überwinden muss, werden die Radstreben so weit wie möglich rotiert so dass der Grundkörper immer parallel zum Boden ausgerichtet ist. Dies vermindert das Risiko umzukippen und sollte eine einache Umgehung von Hindernissen sein, die höher als der Radradius sind. Dieses System ist vergleichbar mit dem Rocker-Bogie Aufhängesystem der Mars-Rover von NASA \cite{rockerbogie}. Wegen des hohen Umsturzrisikos durch den Wind wurde nicht das passive Rocker-Bogie Design gewählt, weil dort keine Möglichkeit des einfachen Aufrichtens besteht.

Octanis' Maximalgeschwindigkeit beträgt schätzungsweise 100 cm/min. Ungeachtet der verfügbaren Menge an Energie ergibt dies eine Tagesreichweite von 1 km. Zwei unterschiedliche Modi werden verwendet: Discovery und Shortest Path. Im \textbf{Discovery Modus} wird der Roboter entweder eine vorprogrammierte Kreisfläche auf der Karte dicht durchfahren, oder bei einer Region mit einem ausserordendlichen Phänomen (z.B. einen hohen pH-Messwert). Der \textbf{Shortest Path Modus} bringt Octanis über den kürzesten Weg von Punkt zu Punkt. Beide Modi können kombiniert verwendet werden. Detektierte Hindernisse können als Markierung gespeichert werden um eine ungefähre Karte zu erstellen.


\subsection{Communication}

Regular data transmission is a mission critical feature of the rover. It is achieved by using the Iridium Short Burst Data (SBD) service with the help of the RockBLOCK Iridium 9602 modem \cite{iridium}. This modem allows the transmission of data packets of up to 340 bytes outwards and 270 bytes inwards approximately every 20 seconds. Larger data frames can be fragmented and transmitted with multiple packets. The key to this modem and network is that is available everywhere on Earth at any time, something that cannot be said of the GSM network or even amateur radio. That the modem require just as low power as the GPS on board adds to the list of benefits. Information sent through the Iridium network is automatically sent to a server and uploaded to the project website, where the public can view Octanis' state and data.

Octanis 1 will also be equipped with a 5 watt one-way APRS transmitter, transmitting sensor data and location information on the 2-meter band. This can be received by amateur radio operators (or unlicensed individuals with radio scanners) who are in range. The range of this transmitter is typically hundreds of kilometers when travelling in the sky (i.e. attached to a balloon). On the ground, the range is orders of magnitude smaller.


\subsection{Optical}
Octanis will be able to take snapshots on-demand with a low resolution camera. It is part of an experiment to find out if an image can be efficiently sent via the Iridium network. Additionally we would like to test on-board image processing (e.g. simple feature detection). The optical system is completed by distance sensors that can detect nearest obstacles. It will be evaluated if the Neato XV-11's LIDAR \cite{lidar} can be used or if we should proceed with IR distance sensors.


\subsection{Scientific Instrument}
The preferred scientific instrument for this mission is a ice core drill combined with a pH probe. The drill is able to drill a small 1cm x 5cm ice/snow core, retract it from the cold surface and then melt the sample. The ice/snow-water can be analysed with a pH probe, that is typically a glass electrode, to determine the pH value. Due to the low-cost nature of the sensor to be used, precision cannot be guaranteed by the manufacturer and has to be calibrated and tested by us. It is though assumed that the device will give good readings for comparison of a high quantity of samples.

\pagebreak

\section{Conclusion}
Knowing that this mission is an ambitious one and will have many challenges to be solved on the way, we believe that it is a possible and viable project. When we can achieve our objectives, we will be able to provide the scientific community (including so-called citizen scientists) with a universal and adaptable rover platform. Proving ourselves in the harsh environment of Antarctica means that our rover can survive many other similar environments. \\

We hope to fascinate and inspire students, teachers, artists, entrepreneurs, scientists - in the end all people - for interdisciplinary research, engineering and knowledge sharing, hoping that they will join us on the grand mission to a better world!


\pagebreak
\pagestyle{empty}
\begin{thebibliography}{1}


\bibitem{krishnakant}
  Krishnakant Babanrao Budhavant, Pasumarthi Surya Prakasa Rao, Pramod Digambar Safai,
  \emph{Chemical Composition of Snow-Water and Scavenging Ratios over Costal Antarctica}.
  Aerosol and Air Quality Research, 14: 666–676, 2014.

\bibitem{octanis}
{\em »octanis | discovery and exploration« website} http://octanis.org, 23.6.2014.

\bibitem{iridium}
{\em Rock7 RockBLOCK website} http://rockblock.rock7mobile.com/products-rockblock.php, 25.6.2014.


\bibitem{octanisgithub}
{\em Octanis 1 GitHub repositories website} http://github.com/octanis1, 25.6.2014.

\bibitem{muses}
  Brian H. Wilcox, Ross M. Jones, 
  \emph{The MUSES-CN Nanorover Mission and Related Technology}.
  IEEE Aerospace 2000 Conference, 18.3.1999.


\bibitem{rockerbogie}
  Hayati, S., et. al., 
  \emph{The Rocky 7 Rover: A Mars Sciencecraft Prototype}.
  Proceedings of the 1997 IEEE International Conference on Robotics and Automation, pp. 2458-64, 1997.

\bibitem{hysplit}
  {\em HYSPLIT - Hybrid Single Particle Lagrangian Integrated Trajectory Model }, NOAA, http://www.ready.noaa.gov/HYSPLIT.php, 23.6.2014.

\bibitem{hysplitjava}
  {\em JAVA software to run multiple HYSPLIT simulations to find the optimal parameters for a trajectory}, http://github.com/octanis1/OctanisHYSPLIT, 23.6.2014.

\bibitem{hysplitexamples}
	{\em Ballooning with HYSPLIT}, \\
	http://www.arl.noaa.gov/documents/workshop/Spring2010/Balloon\_flights.ppt, 23.6.2014.

\bibitem{fablab}
	{\em Fab lab definition}, \\
	http://en.wikipedia.org/wiki/Fab\_lab, 25.6.14.

\bibitem{hackerspace}
	{\em Hackerspace definition}, \\
	http://en.wikipedia.org/wiki/Hackerspace, 25.6.14.

\bibitem{pvedu}
	{\em Effect of Light Intensity}, \\
	http://www.pveducation.org/pvcdrom/solar-cell-operation/effect-of-light-intensity, 25.6.14.

\bibitem{lidar}
	{\em Neato XV-11 LIDAR / Piccolo Laser Distance Sensor}, \\
	http://xv11hacking.wikispaces.com/LIDAR+Sensor, 25.6.14.

\bibitem{solarc}
	{\em Solar constant}, \\
	http://en.wikipedia.org/wiki/Solar\_constant, 25.6.14.

\end{thebibliography}

\end{document}